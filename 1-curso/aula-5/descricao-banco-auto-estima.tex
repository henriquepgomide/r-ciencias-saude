\documentclass[12pt]{article}
\usepackage[utf8]{inputenc}
\usepackage[brazilian]{babel}
\usepackage{url}
\title{\textbf{Curso Ninja em R para Ciências da Saúde} \\ Guia do Banco autoestima.csv}
\author{Henrique Pinto gomide}
\date{}
\begin{document}

\maketitle

\section{Introdução}

O banco de dados "auto-estima.csv" contém 31 variavéis com 56 observações. Ele é possui variáveis de classificação socio-demográfica e do instrumento "Escala de auto-estima de Rosenberg".

Os dados foram coletados através da internet, não sendo feita nenhuma amostragem. Seu fim é para estudo somente.

Link para download: \url{https://www.dropbox.com/s/yg5ah2au41zoeun/autoestima.csv}.

\section{Descrição das variáveis}

\subsection{Sócio-demográfico}

\begin{itemize}
	\item x - Código do questionário
	\item v1 - Data e hora que o questionário foi preenchido
	\item v2 - Nome completo - apagado para evitar identificação
	\item v3 - Idade em anos
	\item v4 e v5 - Email - apagado para evitar identificação
	\item v6 - Sexo - "Masculino" ou "Feminino"
	\item v7 - Estado civil - "Casado(a)" "Divorciado(a)/Separado(a)" "Solteiro(a)"
	\item v8 - Escolaridade - "Ensino Fundamental Completo", "Ensino Médio Completo", "Ensino Médio Incompleto", "Ensino Superior Completo" e "Ensino Superior Incompleto"
	\item v9 - Pratica alguma religião - "Sim", "Não"
\end{itemize}

\subsection{Escala de Auto estima}
O instrumento possui 10 questões, que podem ser respondidas entre quatro categorias. Este valores devem ser convertidos e somados segundo o seguinte critério: "Concordo Plenamente" - 4, "Concordo" - 3, "Discordo" - 2, "Discordo Plenamente" -1.

Abaixo os títulos das questões:
\begin{itemize}
	\item v10 -  De uma forma geral (apesar de tudo), você está satisfeito (a) consigo mesmo (a).
	\item v11 - Às vezes, você acha que não serve para nada (desqualificado (a) ou inferior em relação aos outros).
	\item v12 - Você sente que tem um tanto (um número) de boas qualidades.
	\item v13 - Você é capaz de fazer coisas tão bem quanto a maioria das outras pessoas (desde que te ensinadas).
	\item v14 - Você não sente satisfação nas coisas que realizou. Você sente que não tem muito do que se orgulhar.
	\item v15 - Às vezes, você realmente se sente inútil (incapaz de fazer as coisas).
	\item v16 - Você sente que é uma pessoa de valor, pelo menos num plano igual (num mesmo nível) às outras pessoas.
	\item v17 - Você não se dá o devido valor. Você gostaria de ter mais respeito por você mesmo (a).
	\item v18 - Quase sempre você está inclinado (a) a achar que é um(a) fracassado(a).
	\item v19 - Você tem uma atitude positiva (pensamentos, atos e sentimentos positivos) em relação a você mesmo (a).
\end{itemize}

\subsection{Recodificação do banco e soma da escala}
\begin{itemize}
	\item v10r a v19r - variáveis recodificadas segundo critério estabelecido na sessão anterior.
	\item somaescala - soma da escala de auto-estima. 
\end{itemize}

\end{document}
